\documentclass {article}
\usepackage{fullpage}

\begin{document}

~\vfill
\begin{center}
\Large

{\bf A5 Project Report}

{\bf Title:} Ray traced Backgammon

{\bf Name:} Mahammad Ismayilzada

{\bf Student ID:} 20496519

{\bf User ID:} mismayil
\end{center}
\vfill ~\vfill~
\newpage
\begin{description}
\item[Compile]:\\
The project is compiled as before by invoking make. There are 2 define directives(OPT and AABB) from A4 that can be added to make
which will enable the acceleration and draw only Axis Aligned Bounded Boxes respectively, but OPT is enabled by default and AABB
won't be needed as there no meshes rendered in my project scenes.

\item[Run]:\\
The project is run by invoking ./A5 Assets/lua-file. Produced images are stored in images directory.

\item[External libraries]:\\
I have used perlin noise generating function by Ken Perlin from codermind.com to implement the bump mapping and the source code is in PerlinNoise.{hpp, cpp} files.

\item[Objectives]:\\
I have implemented all my objectives listed on proposal, but could not finish the extra objective motion blur as I mentioned in the proposal. However, to speed up the process
of rendering multithreading has been added as an extra objective. Each of 4 threads is allocated one fourth of the image to render and progress for each thread is printed out while
program is running. You can change the number of threads by simply modifying NUM\_THREADS in util.hpp and building again. Addition to that, all config parameters(such as number of soft shadow
rays, recursive tracing depth etc). are also in util.hpp and can be adjusted accordingly. \\
- Texture mapping only supports png files. To see the adaptive antialiasing, ADAPTIVE\_FALSE\_COLOR should be defined in util.hpp which is undefined by default.\\\\
- {\bf gr.material} command has been extended and overloaded as follows:\\
gr.material(kd, ks) \\
gr.material(kd, ks, shininess) \\
gr.material(kd, ks, shininess, reflectiveness) \\
gr.material(kd, ks, shininess, reflectiveness, refractive\_index) \\
gr.material(kd, ks, shininess, reflectiveness, refractive\_index, transparency) \\
gr.material(kd, ks, shininess, reflectiveness, refractive\_index, transparency, bumpness) \\\\
- New {\bf gr.textmaterial} command has been added and has following constructors: \\
gr.textmaterial('png-file', ks) \\
gr.textmaterial('png-file', ks, shininess) \\
gr.textmaterial('png-file', ks, shininess, reflectiveness) \\
gr.textmaterial('png-file', ks, shininess, reflectiveness, refractive\_index) \\
gr.textmaterial('png-file', ks, shininess, reflectiveness, refractive\_index, transparency) \\
gr.textmaterial('png-file', ks, shininess, reflectiveness, refractive\_index, transparency, bumpness) \\\\
- {\bf gr.arealight} command has been added to support area lights as follows:\\
gr.arealight(light\_pos, light\_colour, attenuation, width, height) where width and height defines the width and height of area light.\\\\
- {\bf gr.dof} command supports Depth of Field:\\
gr.dof(focal\_length, aperture) and gr.render has been extended to accept this node accordingly at the end\\\\
- {\bf gr.joint} command supports CSG nodes:\\
gr.joint(name, op) where op is one of 0=UNION, 1=INTERSECTION or 2=DIFFERENCE and arbitrary number of nodes are added as children of this node\\\\

\end{description}
\end{document}
